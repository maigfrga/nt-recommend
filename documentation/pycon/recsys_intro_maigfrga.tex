Title
Introducción a Sistemas de Recomendación


Elevator Pitch
Los sistemas de recomendación están en todo lado.
Por ejemplo cuando nuestra red social nos invita a conectar con otros usuarios,
o cuando nuestro servicio de música por suscripción nos sugiere nuevo contenido.
Todos estos sistemas comparten algunos principios básicos que revisaremos en esta charla.


Description


Introducción a Sistemas de Recomendación
===========================================

En esta charla analizaremos los aspectos básicos del funcionamiento de los sistemas de recomendación.
La charla se divide en 5 partes:

Introducción (10 minutos)
----------------------------

Empezaremos por discutir brevemente los orígenes y oportunidades de negocio que han generado la expansión
de los sistemas de recomendación. Analizaremos las principales ventajas desde el punto de vista de empresas
y usuarios.

* Un caso del mundo real (El efecto Amazon).

* Anatomía de la cola larga (Long Tail)

* Recomendación basada en contenido.

* Filtrado Colaborativo.

* Sistemas Híbridos.

* Recomendación Conversacional.


nt-recommend: Framework Didáctico Basado en Python3 Para el Aprendizaje de Sistemas de Recomendación (10 minutos)
----------------------------------------------------------------------------------------------------------------------------------------------------------------------------------------------


Filtrado colaborativo. (15 minutos)
----------------------------------------------------------

### Recomendaciones basadas en usuarios.

* Ratings de usuarios.
* Matriz de ratings de usuarios.
* Rating promedio.
* Similitud entre usuarios.
* Vecindarios de similitudes.
* Generando predicciones.
* Pros y cons de filtrado colaborativo basado en usuarios.

### Mejorando computo de similitud.

* Calculo de similitud basado en cosenos.
* Coeficiente de correlación de Pearson.

### Mejorando el computo de predicciones

* Formula de predicción de Resnick
* Ponderación de valores.
* Voto por defecto.

### Predicción de Ratings vs Recomendación de Top N Items


### Evaluación de Sistemas Basados en Filtrado Colaborativo.

* Sub-Sampling Aleatorio (Repeated Random Sub-Sampling).

* Validación Cruzada K-Pliegues (K-Fold Cross Validation).

* Dejar uno fuera (Leave-One-Out).

### Metricas para prediccion.

* Error medio al cuadrado (MSE).

* Raíz cuadrada de error medio al cuadrado (MSE).

* Cobertura.


Conclusiones (5 minutos)
--------------------------

* Fuentes de información.
* Premio Netflix.





Notes

Hola, primero que todo gracias por tener el liderazgo e iniciativa de llevar la pycon en Colombia.
Muy feliz de ver como esta edición es en Medellin, muy bueno descentralizar el conocimiento.
La charla de sistemas de recomendación me parece interesante porque es un tema muy vigente, con muchas
aplicaciones y potencial, a lo mejor hay personas trabajando en este tema en Colombia, pero no he visto
conferencias o meetups sobre el tema aun.
No se requieren grandes conocimientos en python. El código de ejemplo usa python3 y un par de librerías
externas, nada muy complicado.
Los conceptos básicos los he aprendido en una maestría que acabo de terminar en University College Dublin (UCD)
en Irlanda, donde vivo actualmente. Para la fecha de la pycon estaré en Colombia de vacaciones, lo único que
necesitaría es transporte desde el aeropuerto de Armenia, o en su defecto, desde el aeropuerto de Pereira.
Los gastos de alojamiento irían por mi cuenta.

Gracias



Amante de la música y las ciencias de la computación. Ingeniero de sistemas de la Universidad del Quindio.
M.S.C Computer Science University College Dublin (UCD). Actualmente ingeniero líder para instrumentos
musicales fender en Dublin, construyendo software para el aprendizaje musical con mas de 2 millones de
usuarios al mes. Experiencia en sistemas distribuidos, procesamiento de datos en tiempo real, data science,
sistemas de extraccion de informacion musical.
